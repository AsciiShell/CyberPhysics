%!TEX TS-program = xelatex

% Шаблон документа LaTeX создан в 2018 году
% Алексеем Подчезерцевым
% В качестве исходных использованы шаблоны
% 	Данилом Фёдоровых (danil@fedorovykh.ru) 
%		https://www.writelatex.com/coursera/latex/5.2.2
%	LaTeX-шаблон для русской кандидатской диссертации и её автореферата.
%		https://github.com/AndreyAkinshin/Russian-Phd-LaTeX-Dissertation-Template

\documentclass[a4paper,14pt]{article}

\input{data/preambular.tex}
\begin{document} % конец преамбулы, начало документа
\begin{titlepage}
	\begin{center}
		ФЕДЕРАЛЬНОЕ  ГОСУДАРСТВЕННОЕ АВТОНОМНОЕ \\
		ОБРАЗОВАТЕЛЬНОЕ УЧРЕЖДЕНИЕ ВЫСШЕГО ОБРАЗОВАНИЯ\\
		«НАЦИОНАЛЬНЫЙ ИССЛЕДОВАТЕЛЬСКИЙ УНИВЕРСИТЕТ\\
		«ВЫСШАЯ ШКОЛА ЭКОНОМИКИ»
	\end{center}
	
	\begin{center}
		\textbf{Московский институт электроники и математики}
		
		\textbf{Им. А.Н.Тихонова НИУ ВШЭ}
		
		\textbf{Департамент электронной инженерии}
	\end{center}	
	\vspace{5ex}
	\begin{center}
\textbf{<<ПОСТРОЕНИЕ, ПРИМЕНЕНИЕ И ИССЛЕДОВАНИЕ КОМПОНЕНТОВ СИСТЕМ СБОРА ДАННЫХ>>}
	\end{center}	
	\vspace{1ex}
	\begin{center}
\textbf{Отчёт по части 3 лабораторного практикума по дисциплине \\
	<<Электротехника, электроника и метрология>>, раздел <<Метрология>>(ЛР 8-9)}
	\end{center}	
	\vspace{5ex}
	
	\begin{multicols}{2}
	\vfill\null
	\columnbreak
	ВЫПОЛНИЛИ:
	
	Подчезерцев Алексей Евгеньевич
	
	Солодянкин Андрей Александрович
	
	группа БИВ172
	\end{multicols}

	\vfill
	\begin{center}
		Москва \the\year
	\end{center}
\end{titlepage}

\section{Переходные процессы в RLC цепях}

\begin{figure}[H]
	\centering
	\includegraphics[width=0.5\linewidth]{scema.jpg}
	\caption{Схема RLC цепи}	
\end{figure}

\pagebreak
\section{L=10мГн, C=100нФ}

%%% 1

$R_{kr} = 2 * \sqrt{\dfrac{L}{C}} = 632.45$ Ом

\begin{figure}[H]
	\centering
	\includegraphics[width=\linewidth]{1_100.png}
	\includegraphics[width=\linewidth]{sp1_100.png}
	\caption{$R < R_{kr}$}	
\end{figure}

\begin{figure}[H]
	\centering
	\includegraphics[width=\linewidth]{1_700.png}
	\includegraphics[width=\linewidth]{sp1_700.png}
	\caption{$R > R_{kr}$}	
\end{figure}

\pagebreak
\section{L=10мГн, C=22нФ}

%%% 2

$R_{kr} = 2 * \sqrt{\dfrac{L}{C}} = 1348.40$ Ом

\begin{figure}[H]
	\centering
	\includegraphics[width=\linewidth]{2_100.png}
	\includegraphics[width=\linewidth]{sp2_100.png}
	\caption{$R < R_{kr}$}	
\end{figure}

\begin{figure}[H]
	\centering
	\includegraphics[width=\linewidth]{2_1400.png}
	\includegraphics[width=\linewidth]{sp2_1400.png}
	\caption{$R > R_{kr}$}	
\end{figure}

\pagebreak
\section{L=1мГн, C=22нФ}

%%% 3

$R_{kr} = 2 * \sqrt{\dfrac{L}{C}} = 200$ Ом

\begin{figure}[H]
	\centering
	\includegraphics[width=\linewidth]{3_100.png}
	\includegraphics[width=\linewidth]{sp3_100.png}
	\caption{$R < R_{kr}$}	
\end{figure}

\begin{figure}[H]
	\centering
	\includegraphics[width=\linewidth]{3_2100.png}
	\includegraphics[width=\linewidth]{sp3_2100.png}
	\caption{$R > R_{kr}$}	
\end{figure}


\end{document} % конец документа

