%!TEX TS-program = xelatex

% Шаблон документа LaTeX создан в 2018 году
% Алексеем Подчезерцевым
% В качестве исходных использованы шаблоны
% 	Данилом Фёдоровых (danil@fedorovykh.ru) 
%		https://www.writelatex.com/coursera/latex/5.2.2
%	LaTeX-шаблон для русской кандидатской диссертации и её автореферата.
%		https://github.com/AndreyAkinshin/Russian-Phd-LaTeX-Dissertation-Template

\documentclass[a4paper,14pt]{article}

\input{data/preambular.tex}
\begin{document} % конец преамбулы, начало документа
\begin{titlepage}
	\begin{center}
		ФЕДЕРАЛЬНОЕ  ГОСУДАРСТВЕННОЕ АВТОНОМНОЕ \\
		ОБРАЗОВАТЕЛЬНОЕ УЧРЕЖДЕНИЕ ВЫСШЕГО ОБРАЗОВАНИЯ\\
		«НАЦИОНАЛЬНЫЙ ИССЛЕДОВАТЕЛЬСКИЙ УНИВЕРСИТЕТ\\
		«ВЫСШАЯ ШКОЛА ЭКОНОМИКИ»
	\end{center}
	
	\begin{center}
		\textbf{Московский институт электроники и математики}
		
		\textbf{Им. А.Н.Тихонова НИУ ВШЭ}
		
		\textbf{Департамент электронной инженерии}
	\end{center}	
	\vspace{5ex}
	\begin{center}
\textbf{<<ПОСТРОЕНИЕ, ПРИМЕНЕНИЕ И ИССЛЕДОВАНИЕ КОМПОНЕНТОВ СИСТЕМ СБОРА ДАННЫХ>>}
	\end{center}	
	\vspace{1ex}
	\begin{center}
\textbf{Отчёт по части 3 лабораторного практикума по дисциплине \\
	<<Электротехника, электроника и метрология>>, раздел <<Метрология>>(ЛР 8-9)}
	\end{center}	
	\vspace{5ex}
	
	\begin{multicols}{2}
	\vfill\null
	\columnbreak
	ВЫПОЛНИЛИ:
	
	Подчезерцев Алексей Евгеньевич
	
	Солодянкин Андрей Александрович
	
	группа БИВ172
	\end{multicols}

	\vfill
	\begin{center}
		Москва \the\year
	\end{center}
\end{titlepage}
\tableofcontents
\pagebreak

\section{Задание}

\begin{figure}[H]
	\centering		
	\includegraphics[width=0.4\linewidth]{image/schema_task}
	\caption{Схема задания}\label{img:schema_task}
\end{figure}

\section{Параметры схемы}

\begin{table}[H]	
	\begin{center}
	\begin{flushleft}
		\tablecaption{Топологические и геометрические нормы}
	\end{flushleft}
	\begin{tabular}{|l|l|l|}

		\hline
		Δ        & Минимальный топологический размер, мкм                                               & 0.25     \\ \hline
		& Минимальный размер стороны контактного окна                                          & 0.5Δ     \\ \hline
		& Минимальный размер фигуры в слое металла                                             & 0.5Δ     \\ \hline
		& Минимальное перекрытие металлом контактного окна                                     & 0.5Δ     \\ \hline
		& Минимальное расстояние от границ контактного   & 0.5Δ     \\ 
		& до границ контактируемой области   &     \\ \hline
		& Минимальное расстояние между фигурами в слое металла                                 & 0.5Δ     \\ \hline
		& Остальные топологические нормы                                                       & 0.5Δ     \\ \hline \hline
		
		$R_{sb}$         & Поверхностное сопротивление пассивной области базы, Ом                      & 300      \\ \hline
		$R_{sbb}$        & Поверхностное сопротивление активной области базы, Ом                       & 3000     \\ \hline
		$R_{se}$         & Поверхностное сопротивление области эмитера, Ом/□                           &     10     \\ \hline
		$C_{middle}$     & Ёмкость межсоединения, пФ                                                   &      10    \\ \hline
		$X_{jk}$         & глубина залегания p-n перехода базаколлектор, м                             & $1.1*10^{–6}$ \\ \hline
		$X_{je}$         & глубина залегания p-n перехода базаэмиттер, м                               & $0.5*10^{–6}$ \\ \hline
		$\omega_{epi}$   & толщина эпитаксиального слоя, м                                             & $4*10^{–6}  $ \\ \hline
		$X_{jn}$         & толщина скрытого n+ слоя, м                                                 & $1.0*10^{–6}$ \\ \hline
		$C_{pn0}$        & удельная ёмкость p-n-перехода, Ф/м2                                         & $3*10^{–3}  $ \\ \hline
		\hline
		
		Vdd      & Напряжение питания, В                                                       & $2.5     $ \\ \hline
		Смеж     & Ёмкость межсоединения, пФ                                                   & 10 \\ \hline
		tox      & Толщина подзатворного оксида, нм                                            & $6       $ \\ \hline
		μn       & Подвижность электронов, м2/(Вс)                                             & $0.0317  $ \\ \hline
		μp       & Подвижность дырок, м2/(Вс)                                                  & $0.0136  $ \\ \hline
		dпер     & Перекрытие затвором областей стока/истока, мкм                              & 0.1Δ    \\ \hline
		xj       & Глубина залегания p-n перехода нм                                           & 100 \\ 
		& исток-подложка и сток-подложка                                              & \\ \hline
		L        & длина канала, м                                                             & 1.0Δ     \\ \hline
	\end{tabular}
	\end{center}
\end{table}


\section{Работа транзисторов}

Рассмотрим 27 набор данных, который соответствует следующей бинарной последовательности: $0	0	0	1	1	0	1	1$. При подаче такого сигнала на схему на каждый элемент $AND$ будут поданы разные значения. Рассмотрим их.

\subsection{Первый элемент $AND$}

Элемент $AND$ изображён на \imref{img:schema_and}

\begin{figure}[H]
	\centering		
	\includegraphics[width=\linewidth]{image/schema_and}
	\caption{Схема элемента И}\label{img:schema_and}
\end{figure}

На вход подаются значения $00$, транзисторы М1 и М2 остаются открытыми, а М3 и М4 закрытыми, на инвентор подаётся логическая 1, транзистор М6 закрыт, а М5 открыт, на выходе схемы получаем 0.

\subsection{Второй элемент $AND$}

На вход подаются значения $01$, транзисторы М1 и М4 остаются открытыми, а М3 и М2 закрытыми, на инвентор подаётся логическая 1, транзистор М6 закрыт, а М5 открыт, на выходе схемы получаем 0.

\subsection{Третий элемент $AND$}

На вход подаются значения $10$, транзисторы М2 и М3 остаются открытыми, а М1 и М4 закрытыми, на инвентор подаётся логическая 1, транзистор М6 закрыт, а М5 открыт, на выходе схемы получаем 0.

\subsection{Четвёртый элемент $AND$}

На вход подаются значения $11$, транзисторы М3 и М4 остаются открытыми, а М1 и М2 закрытыми, на инвентор подаётся логический 0, транзистор М6 открыт, а М5 закрыт, на выходе схемы получаем 1.

\subsection{Элемент $OR4$}

Элемент $OR4$ изображён на \imref{img:schema_or}

\begin{figure}[H]
	\centering		
	\includegraphics[width=\linewidth]{image/schema_or}
	\caption{Схема элемента 4-ИЛИ}\label{img:schema_or}
\end{figure}

На вход подаются значения $0001$, транзисторы М1, М2, М3 и М5 остаются открытыми, а М4, М6, М7 и М8	 закрытыми, на инвентор подаётся логический 0, транзистор М9 открыт, а М10 закрыт, на выходе схемы получаем 1.

TODO описать все комбинаиции для И-4
\section{Тут идёт кусок теории}

\section{Расчёт параметров схемы}

Для удобства проектирования схемы в LTSpice были созданы переменные, некоторые из которых зависят от исходных

$ Vol = 2.5V  $ -- Напряжение питания

$ un = 0.0317 $ -- Подвижность электронов

$ up = 0.0136 $ -- Подвижность дырок

$ LChan=0.25u $ -- Длина канала

$ WtoL=8 $ -- Выбор отношения ширины канала к длине

$ Depth=3.75u $ -- Глубина канала по построению

$ WN=WtoL * LChan $ -- Ширина канала для NMOPT

$ WP= \frac{un}{up} * WtoL* LChan $ -- Ширина канала для PMOPT

$ ASN=WN * Depth $ -- Площадь стока для NMOPT

$ ADN= ASN $-- Площадь истока для NMOPT

$ ASP=WP * Depth$ -- Площадь стока для PMOPT

$ ADP=ASP $ -- Площадь стока для PMOPT

$ PSN=2 * WN + Depth $ -- Периметр истока для NMOPT

$ PDN=2 * WN + Depth $ -- Периметр стока для NMOPT

$ PSP=2 * WP + Depth $ -- Периметр истока для PMOPT

$ PDP=2 * WP + Depth $ -- Периметр стока для PMOPT


\section{Вид сверху и слайсы}

\section{Spice calculator}

Схемы элементов представлены на \imref{img:schema_and} и \imref{img:schema_or}

\subsection{Таблица истинности}

\begin{figure}[H]
	\centering		
	\includegraphics[width=\linewidth]{image/schema_all}
	\caption{Схема для проверки таблицы истинности}\label{img:schema_all}
\end{figure}


\begin{figure}[H]
	\centering		
	\includegraphics[width=\linewidth]{image/spice_and_bin}
	\caption{Таблица истинности для элемента И}\label{img:spice_and_bin}
\end{figure}


\begin{figure}[H]
	\centering		
	\includegraphics[width=\linewidth]{image/spice_or_bin}
	\caption{Таблица истинности для элемента-ИЛИ }\label{img:spice_or_bin}
\end{figure}

\begin{figure}[H]
	\centering		
	\includegraphics[width=\linewidth]{image/spice_bin}
	\caption{Таблица истинности всей схемы}\label{img:spice_bin}
\end{figure}

\subsection{Анализ статических характеристик}

\subsection{Анализ динамических характеристик}

\begin{figure}[H]
	\centering
	\includegraphics[width=0.7\linewidth]{image/dyn_shema_min}
	\caption{Схема расчета динамических характеристик}
	\label{fig:dynshema}
\end{figure}

\begin{figure}[H]
	\centering
	\includegraphics[width=\linewidth]{image/dyn_max_freq}
	\caption{}
	\label{fig:dyn_max_freq}
\end{figure}

\begin{figure}[H]
	\centering
	\includegraphics[width=\linewidth]{image/dyn_max_volt}
	\caption{}
	\label{fig:dyn_max_volt}
\end{figure}

$$T(f_{max}) = 27ns$$
$$f_{max} = \dfrac{1}{T_{max}} \approx 37.037 MHz$$

\begin{figure}[H]
	\centering
	\includegraphics[width=0.7\linewidth]{image/dyn_shema_opt}
	\caption{}
	\label{fig:dynshemamin}
\end{figure}

\begin{figure}[H]
	\centering
	\includegraphics[width=\linewidth]{image/dyn_opt_zad01}
	\caption{}
	\label{fig:dynoptzad01}
\end{figure}


\begin{figure}[H]
	\centering
	\includegraphics[width=\linewidth]{image/dyn_opt_zad10}
	\caption{}
	\label{fig:dynoptzad10}
\end{figure}

$$t_{z}^{01} = 12ns$$
$$t_{z}^{10} = 14ns$$
$$t_{z} = \dfrac{t_{z}^{01} + t_{z}^{10}}{2} = 13ns$$

\begin{figure}[H]
	\centering
	\includegraphics[width=\linewidth]{image/dyn_opt_f01}
	\caption{}
	\label{fig:dynoptf01}
\end{figure}

\begin{figure}[H]
	\centering
	\includegraphics[width=\linewidth]{image/dyn_opt_f10}
	\caption{}
	\label{fig:dynoptf10}
\end{figure}

$$$t_{f}^{01} = 25ns$
$$$t_{f}^{10} = 21ns$
$$f_{p} = \dfrac{1}{T} = $$




\section{Топология}

\begin{landscape}
\begin{figure}[H]
	\centering		
	\includegraphics[width=\linewidth]{image/ledit_schema}
	\caption{Топология}\label{img:ledit_schema}
\end{figure}
\end{landscape}

\section{Вывод}

Всё тлен

\end{document} % конец документа