%!TEX TS-program = xelatex

% Шаблон документа LaTeX создан в 2018 году
% Алексеем Подчезерцевым
% В качестве исходных использованы шаблоны
% 	Данилом Фёдоровых (danil@fedorovykh.ru) 
%		https://www.writelatex.com/coursera/latex/5.2.2
%	LaTeX-шаблон для русской кандидатской диссертации и её автореферата.
%		https://github.com/AndreyAkinshin/Russian-Phd-LaTeX-Dissertation-Template

\documentclass[a4paper,14pt]{article}

\input{data/preambular.tex}
\begin{document} % конец преамбулы, начало документа
\begin{titlepage}
	\begin{center}
		ФЕДЕРАЛЬНОЕ  ГОСУДАРСТВЕННОЕ АВТОНОМНОЕ \\
		ОБРАЗОВАТЕЛЬНОЕ УЧРЕЖДЕНИЕ ВЫСШЕГО ОБРАЗОВАНИЯ\\
		«НАЦИОНАЛЬНЫЙ ИССЛЕДОВАТЕЛЬСКИЙ УНИВЕРСИТЕТ\\
		«ВЫСШАЯ ШКОЛА ЭКОНОМИКИ»
	\end{center}
	
	\begin{center}
		\textbf{Московский институт электроники и математики}
		
		\textbf{Им. А.Н.Тихонова НИУ ВШЭ}
		
		\textbf{Департамент электронной инженерии}
	\end{center}	
	\vspace{5ex}
	\begin{center}
\textbf{<<ПОСТРОЕНИЕ, ПРИМЕНЕНИЕ И ИССЛЕДОВАНИЕ КОМПОНЕНТОВ СИСТЕМ СБОРА ДАННЫХ>>}
	\end{center}	
	\vspace{1ex}
	\begin{center}
\textbf{Отчёт по части 3 лабораторного практикума по дисциплине \\
	<<Электротехника, электроника и метрология>>, раздел <<Метрология>>(ЛР 8-9)}
	\end{center}	
	\vspace{5ex}
	
	\begin{multicols}{2}
	\vfill\null
	\columnbreak
	ВЫПОЛНИЛИ:
	
	Подчезерцев Алексей Евгеньевич
	
	Солодянкин Андрей Александрович
	
	группа БИВ172
	\end{multicols}

	\vfill
	\begin{center}
		Москва \the\year
	\end{center}
\end{titlepage}
\tableofcontents
\pagebreak

\section{Задание}

\section{Выполнение работы}

\subsection{Создание базы данных}

\section{Работа транзисторов}

Рассмотрим 27 набор данных, который соответствует следующей бинарной последовательности: $0	0	0	1	1	0	1	1$. При подаче такого сигнала на схему на каждый элемент $AND$ будут поданы разные значения. Рассмотрим их.

	\subsection{Первый элемент $AND$}
	
	Элемент $AND$ изображён на \imref{img:schema_and}
		
	\begin{figure}[H]
		\centering		
		\includegraphics[width=\linewidth]{image/schema_and}
		\caption{Схема элемента И}\label{img:schema_and}
	\end{figure}
	
	На вход подаются значения $00$, транзисторы М1 и М2 остаются открытыми, а М3 и М4 закрытыми, на инвентор подаётся логическая 1, транзистор М6 закрыт, а М5 открыт, на выходе схемы получаем 0.

	\subsection{Второй элемент $AND$}

	На вход подаются значения $01$, транзисторы М1 и М4 остаются открытыми, а М3 и М2 закрытыми, на инвентор подаётся логическая 1, транзистор М6 закрыт, а М5 открыт, на выходе схемы получаем 0.
	
	\subsection{Третий элемент $AND$}
	
	На вход подаются значения $10$, транзисторы М2 и М3 остаются открытыми, а М1 и М4 закрытыми, на инвентор подаётся логическая 1, транзистор М6 закрыт, а М5 открыт, на выходе схемы получаем 0.

	\subsection{Четвёртый элемент $AND$}

	На вход подаются значения $11$, транзисторы М3 и М4 остаются открытыми, а М1 и М2 закрытыми, на инвентор подаётся логический 0, транзистор М6 открыт, а М5 закрыт, на выходе схемы получаем 1.
	
	\subsection{Элемент $OR4$}
	
	Элемент $OR4$ изображён на \imref{img:schema_or}
	
	\begin{figure}[H]
		\centering		
		\includegraphics[width=\linewidth]{image/schema_or}
		\caption{Схема элемента 4-ИЛИ}\label{img:schema_or}
	\end{figure}

	На вход подаются значения $0001$, транзисторы М1, М2, М3, М5 остаются открытыми, а М1 и М2 закрытыми, на инвентор подаётся логический 0, транзистор М6 открыт, а М5 закрыт, на выходе схемы получаем 1.
	
\section{Вывод}



\end{document} % конец документа