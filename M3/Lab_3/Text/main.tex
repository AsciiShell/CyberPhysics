%!TEX TS-program = xelatex

% Шаблон документа LaTeX создан в 2018 году
% Алексеем Подчезерцевым
% В качестве исходных использованы шаблоны
% 	Данилом Фёдоровых (danil@fedorovykh.ru) 
%		https://www.writelatex.com/coursera/latex/5.2.2
%	LaTeX-шаблон для русской кандидатской диссертации и её автореферата.
%		https://github.com/AndreyAkinshin/Russian-Phd-LaTeX-Dissertation-Template

\documentclass[a4paper,14pt]{article}

\input{data/preambular.tex}
\begin{document} % конец преамбулы, начало документа
\begin{titlepage}
	\begin{center}
		ФЕДЕРАЛЬНОЕ  ГОСУДАРСТВЕННОЕ АВТОНОМНОЕ \\
		ОБРАЗОВАТЕЛЬНОЕ УЧРЕЖДЕНИЕ ВЫСШЕГО ОБРАЗОВАНИЯ\\
		«НАЦИОНАЛЬНЫЙ ИССЛЕДОВАТЕЛЬСКИЙ УНИВЕРСИТЕТ\\
		«ВЫСШАЯ ШКОЛА ЭКОНОМИКИ»
	\end{center}
	
	\begin{center}
		\textbf{Московский институт электроники и математики}
		
		\textbf{Им. А.Н.Тихонова НИУ ВШЭ}
		
		\textbf{Департамент электронной инженерии}
	\end{center}	
	\vspace{5ex}
	\begin{center}
\textbf{<<ПОСТРОЕНИЕ, ПРИМЕНЕНИЕ И ИССЛЕДОВАНИЕ КОМПОНЕНТОВ СИСТЕМ СБОРА ДАННЫХ>>}
	\end{center}	
	\vspace{1ex}
	\begin{center}
\textbf{Отчёт по части 3 лабораторного практикума по дисциплине \\
	<<Электротехника, электроника и метрология>>, раздел <<Метрология>>(ЛР 8-9)}
	\end{center}	
	\vspace{5ex}
	
	\begin{multicols}{2}
	\vfill\null
	\columnbreak
	ВЫПОЛНИЛИ:
	
	Подчезерцев Алексей Евгеньевич
	
	Солодянкин Андрей Александрович
	
	группа БИВ172
	\end{multicols}

	\vfill
	\begin{center}
		Москва \the\year
	\end{center}
\end{titlepage}
\tableofcontents
\pagebreak
\section{Краткое содержание работы}

В работе исследуются статические и динамические характеристики симметричного триггера на биполярных транзисторах.

Работа состоит из двух частей.

В первой части изучается статический режим работы на постоянном токе,
проверяется возможность триггера сохранять свое состояние длительное время,
определяются уровни логической единицы и логического нуля.

Во второй части работы изучаются динамические характеристики триггера,
исследуются формы сигналов на выходах триггера и их зависимость от параметров схемы.

\section{Теоретическое введение}


\section{Описание установки}


\section{Результаты измерений}

\begin{table}[H]
	\begin{center}
		\tablecaption{Таблица состояний Т триггера}
	\begin{tabular}{|l|l|l|l|l|l|l|l|}
		\hline
		&         & V3    & V13   & V6   & V7   & V10  & Ir, mA \\ \hline
		Q=1 & S=1 R=0 & 2.3   & 11.19 & 0.7  & 2.9  & 2.2  & 0.818  \\ \hline
		& S=0 R=0 & 2.1   & 11.18 & 0.7  & 2.7  & 2    & 0.833  \\ \hline
		Q=0 & S=0 R=1 & 11    & 2.57  & 3.2  & 0.9  & 2.5  & 0.166  \\ \hline
		& S=0 R=0 & 11.04 & 2.47  & 3.06 & 0.86 & 2.34 & 0.159  \\ \hline
		& S=1 R=1 & 3.84  & 3.83  & 1.26 & 1.33 & 3.77 & 0.247  \\ \hline
	\end{tabular}
\end{center}
\end{table}






\end{document} % конец документа