%!TEX TS-program = xelatex

% Шаблон документа LaTeX создан в 2018 году
% Алексеем Подчезерцевым
% В качестве исходных использованы шаблоны
% 	Данилом Фёдоровых (danil@fedorovykh.ru) 
%		https://www.writelatex.com/coursera/latex/5.2.2
%	LaTeX-шаблон для русской кандидатской диссертации и её автореферата.
%		https://github.com/AndreyAkinshin/Russian-Phd-LaTeX-Dissertation-Template

\documentclass[a4paper,14pt]{article}

\input{data/preambular.tex}
\begin{document} % конец преамбулы, начало документа
\begin{titlepage}
	\begin{center}
		ФЕДЕРАЛЬНОЕ  ГОСУДАРСТВЕННОЕ АВТОНОМНОЕ \\
		ОБРАЗОВАТЕЛЬНОЕ УЧРЕЖДЕНИЕ ВЫСШЕГО ОБРАЗОВАНИЯ\\
		«НАЦИОНАЛЬНЫЙ ИССЛЕДОВАТЕЛЬСКИЙ УНИВЕРСИТЕТ\\
		«ВЫСШАЯ ШКОЛА ЭКОНОМИКИ»
	\end{center}
	
	\begin{center}
		\textbf{Московский институт электроники и математики}
		
		\textbf{Им. А.Н.Тихонова НИУ ВШЭ}
		
		\textbf{Департамент электронной инженерии}
	\end{center}	
	\vspace{5ex}
	\begin{center}
\textbf{<<ПОСТРОЕНИЕ, ПРИМЕНЕНИЕ И ИССЛЕДОВАНИЕ КОМПОНЕНТОВ СИСТЕМ СБОРА ДАННЫХ>>}
	\end{center}	
	\vspace{1ex}
	\begin{center}
\textbf{Отчёт по части 3 лабораторного практикума по дисциплине \\
	<<Электротехника, электроника и метрология>>, раздел <<Метрология>>(ЛР 8-9)}
	\end{center}	
	\vspace{5ex}
	
	\begin{multicols}{2}
	\vfill\null
	\columnbreak
	ВЫПОЛНИЛИ:
	
	Подчезерцев Алексей Евгеньевич
	
	Солодянкин Андрей Александрович
	
	группа БИВ172
	\end{multicols}

	\vfill
	\begin{center}
		Москва \the\year
	\end{center}
\end{titlepage}
\tableofcontents
\pagebreak
\section{Цели работы}
\begin{itemize}
	\item Экспериментальное Исследование входных и выходных вольт-амперных характеристик (ВАХ) биполярного транзистора.
	
	\item Приобретение навыков измерения характеристик биполярных транзисторов и  определения основных параметров схемотехнической модели Гуммеля-Пуна транзистора по резкльтатам измерения его ВАХ.
	
	\item Приобретение навыков расчета схем и моднлирование характеристик транзисторов с помощью программы схемотехнического моделированря SPICE.
\end{itemize}

\section{Теоретическое введение}

\begin{figure}[H]
	\centering
	\includegraphics[width=\linewidth]{images/theory_1}
	\caption*{}
	\label{fig:theory1}
\end{figure}

\begin{figure}[H]
	\centering
	\includegraphics[width=\linewidth]{images/theory_2}
	\caption*{}
	\label{fig:theory2}
\end{figure}

\begin{figure}[H]
	\centering
	\includegraphics[width=\linewidth]{images/theory_3}
	\caption*{}
	\label{fig:theory3}
\end{figure}

\begin{figure}[H]
	\centering
	\includegraphics[width=\linewidth]{images/theory_4}
	\caption*{}
	\label{fig:theory4}
\end{figure}

\begin{figure}[H]
	\centering
	\includegraphics[width=\linewidth]{images/theory_5}
	\caption*{}
\end{figure}

\begin{figure}[H]
	\centering
	\includegraphics[width=\linewidth]{images/theory_6}
	\caption*{}
\end{figure}

\begin{figure}[H]
	\centering
	\includegraphics[width=\linewidth]{images/theory_7}
	\caption*{}
\end{figure}

\pagebreak

\section{Схемы измерений}

\section{Таблицы с результатами измерений}
\subsection{ Задание 1}
	\begin{table}[H]
		\begin{tabular}{|l|l|l|}
			\hline
			\multicolumn{1}{|c|}{Uб} & \multicolumn{1}{c|}{Iб} & \multicolumn{1}{c|}{Iк} \\ \hline
			0.761                    & 1.94                    & 10.6                    \\ \hline
			0.758                    & 1.82                    & 10.59                   \\ \hline
			0.752                    & 1.35                    & 10.58                   \\ \hline
			0.749                    & 1.12                    & 10.57                   \\ \hline
			0.744                    & 0.92                    & 10.56                   \\ \hline
			0.738                    & 0.6                     & 10.52                   \\ \hline
			0.734                    & 0.42                    & 10.51                   \\ \hline
			0.732                    & 0.36                    & 10.5                    \\ \hline
			0.73                     & 0.3                     & 10.48                   \\ \hline
			0.725                    & 0.25                    & 10.46                   \\ \hline
			0.721                    & 0.2                     & 10.41                   \\ \hline
			0.718                    & 0.15                    & 10.32                   \\ \hline
			0.7                      & 0.12                    & 8.09                    \\ \hline
			0.668                    & 0.08                    & 3.19                    \\ \hline
			0.639                    & 0.06                    & 1.12                    \\ \hline
			0.6                      & 0.05                    & 0.24                    \\ \hline
			0.485                    & 0.05                    & 0.003                   \\ \hline
			0.421                    & 0.05                    & 0.0008                  \\ \hline
			0.334                    & 0.05                    & 0.0006                  \\ \hline
			0.033                    & 0.05                    & 0.0006                  \\ \hline
		\end{tabular}
	\end{table}

\end{document} % конец документа

