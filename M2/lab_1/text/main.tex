%!TEX TS-program = xelatex

% Шаблон документа LaTeX создан в 2018 году
% Алексеем Подчезерцевым
% В качестве исходных использованы шаблоны
% 	Данилом Фёдоровых (danil@fedorovykh.ru) 
%		https://www.writelatex.com/coursera/latex/5.2.2
%	LaTeX-шаблон для русской кандидатской диссертации и её автореферата.
%		https://github.com/AndreyAkinshin/Russian-Phd-LaTeX-Dissertation-Template

\documentclass[a4paper,14pt]{article}

\input{data/preambular.tex}
\begin{document} % конец преамбулы, начало документа
\begin{titlepage}
	\begin{center}
		ФЕДЕРАЛЬНОЕ  ГОСУДАРСТВЕННОЕ АВТОНОМНОЕ \\
		ОБРАЗОВАТЕЛЬНОЕ УЧРЕЖДЕНИЕ ВЫСШЕГО ОБРАЗОВАНИЯ\\
		«НАЦИОНАЛЬНЫЙ ИССЛЕДОВАТЕЛЬСКИЙ УНИВЕРСИТЕТ\\
		«ВЫСШАЯ ШКОЛА ЭКОНОМИКИ»
	\end{center}
	
	\begin{center}
		\textbf{Московский институт электроники и математики}
		
		\textbf{Им. А.Н.Тихонова НИУ ВШЭ}
		
		\textbf{Департамент электронной инженерии}
	\end{center}	
	\vspace{5ex}
	\begin{center}
\textbf{<<ПОСТРОЕНИЕ, ПРИМЕНЕНИЕ И ИССЛЕДОВАНИЕ КОМПОНЕНТОВ СИСТЕМ СБОРА ДАННЫХ>>}
	\end{center}	
	\vspace{1ex}
	\begin{center}
\textbf{Отчёт по части 3 лабораторного практикума по дисциплине \\
	<<Электротехника, электроника и метрология>>, раздел <<Метрология>>(ЛР 8-9)}
	\end{center}	
	\vspace{5ex}
	
	\begin{multicols}{2}
	\vfill\null
	\columnbreak
	ВЫПОЛНИЛИ:
	
	Подчезерцев Алексей Евгеньевич
	
	Солодянкин Андрей Александрович
	
	группа БИВ172
	\end{multicols}

	\vfill
	\begin{center}
		Москва \the\year
	\end{center}
\end{titlepage}
\tableofcontents
\pagebreak
\section{Цели работы}
\begin{itemize}
	\item Экспериментальное Исследование входных и выходных вольт-амперных характеристик (ВАХ) биполярного транзистора.
	
	\item Приобретение навыков измерения характеристик биполярных транзисторов и  определения основных параметров схемотехнической модели Гуммеля-Пуна транзистора по резкльтатам измерения его ВАХ.
	
	\item Приобретение навыков расчета схем и моднлирование характеристик транзисторов с помощью программы схемотехнического моделированря SPICE.
\end{itemize}

\section{Теоретическое введение}

\begin{figure}[H]
	\centering
	\includegraphics[width=\linewidth]{images/theory_1}
	\caption*{}
	\label{fig:theory1}
\end{figure}

\begin{figure}[H]
	\centering
	\includegraphics[width=\linewidth]{images/theory_2}
	\caption*{}
	\label{fig:theory2}
\end{figure}

\begin{figure}[H]
	\centering
	\includegraphics[width=\linewidth]{images/theory_3}
	\caption*{}
	\label{fig:theory3}
\end{figure}

\begin{figure}[H]
	\centering
	\includegraphics[width=\linewidth]{images/theory_4}
	\caption*{}
	\label{fig:theory4}
\end{figure}

\begin{figure}[H]
	\centering
	\includegraphics[width=\linewidth]{images/theory_5}
	\caption*{}
\end{figure}

\begin{figure}[H]
	\centering
	\includegraphics[width=\linewidth]{images/theory_6}
	\caption*{}
\end{figure}

\begin{figure}[H]
	\centering
	\includegraphics[width=\linewidth]{images/theory_7}
	\caption*{}
\end{figure}

\pagebreak

\section{Схемы измерений}
\begin{figure}[H]
	\centering
	\includegraphics[width=0.7\linewidth]{images/shema}
	\caption{}
	\label{fig:shema}
\end{figure}

\section{Таблицы с результатами измерений}
\subsection{ Задание 1}
\begin{table}[H]
	\begin{tabular}{|c|c|c|}
		\hline
		Uб(В) & Iб(мкА) & Iк(мкА) \\ \hline
		-0.08 & 0.1     & 2.7     \\ \hline
		-0.03 & 0.1     & 4       \\ \hline
		-0.03 & 0.2     & 7.2     \\ \hline
		0.03  & 0.3     & 16.6    \\ \hline
		0.06  & 0.5     & 24.2    \\ \hline
		0.13  & 0.9     & 53.2    \\ \hline
		0.19  & 1.1     & 70.5    \\ \hline
		0.33  & 1.4     & 94      \\ \hline
		0.48  & 2.2     & 150     \\ \hline
		0.52  & 2.6     & 170     \\ \hline
		0.58  & 5       & 400     \\ \hline
		0.64  & 21.4    & 2600    \\ \hline
		0.68  & 156     & 4901    \\ \hline
		0.71  & 655     & 8420    \\ \hline
	\end{tabular}
\end{table}


\subsection{ Задание 2}

\begin{table}[H]
	\begin{tabular}{|c|c|c|c|c|c|}
		\hline
		\multicolumn{2}{|c|}{Iб = 30мкА} & \multicolumn{2}{c|}{Iб = 50мкА} & \multicolumn{2}{c|}{Iб = 70мкА} \\ \hline
		Uк(В)          & Iк(мА)1         & Uк(В)         & Iк(мА)2         & Uк(В)         & Iк(мА)3         \\ \hline
		0.02           & 0               & 0.01          & 0               & 0.01          & 0               \\ \hline
		0.06           & 0.28            & 0.05          & 0.28            & 0.02          & 0.1124          \\ \hline
		0.09           & 0.73            & 0.09          & 1.25            & 0.05          & 0.378           \\ \hline
		0.11           & 1.019           & 0.28          & 6.15            & 0.08          & 1.5             \\ \hline
		0.15           & 1.69            & 0.99          & 6.73            & 0.1           & 2.55            \\ \hline
		0.29           & 3.39            & 1.74          & 6.93            & 0.13          & 4.16            \\ \hline
		0.69           & 3.76            &               &                 & 0.17          & 6.26            \\ \hline
		1              & 3.86            &               &                 & 0.19          & 7.32            \\ \hline
		1.2            & 3.9             &               &                 & 0.5           & 9.2             \\ \hline
		1.5            & 3.95            &               &                 & 0.58          & 9.25            \\ \hline
		1.75           & 4               &               &                 & 0.6           & 9.23            \\ \hline
		1.95           & 4.02            &               &                 &               &                 \\ \hline
		2.23           & 4.08            &               &                 &               &                 \\ \hline
		2.65           & 4.15            &               &                 &               &                 \\ \hline
		3              & 4.19            &               &                 &               &                 \\ \hline
		3.1            & 4.2             &               &                 &               &                 \\ \hline
	\end{tabular}
\end{table}



\section{Графики}

\begin{figure}[H]
	\centering
	\includegraphics[width=0.7\linewidth]{images/graf11}
	\caption{}
	\label{fig:graf11}
\end{figure}


\begin{figure}[H]
	\centering
	\includegraphics[width=0.7\linewidth]{images/graf3}
	\caption{}
	\label{fig:graf3}
\end{figure}


\begin{figure}[H]
	\centering
	\includegraphics[width=0.7\linewidth]{images/graf1}
	\caption{}
	\label{fig:graf1}
\end{figure}


\section{ Ход и результаты вычисления предварительных значений параметров модели Гуммеля-Пуна по результатам измерений}

$$Rc = \dfrac{0.19}{7.32} * 10^{-3} = 2.596 * 10^{-5} = 259.6Om$$

$$VAF = \dfrac{(3.1 - 0.29)}{4.2 - 3.39} * 3.39 - 0.29 = 12.05B$$ 
 
$$I_{k1} = Is * (exp(\dfrac{U_{be1}}{ nf * fi}) - 1)$$

$$I_{k2} = Is * (exp(\dfrac{U_{be2}}{ nf * fi}) - 1)$$

$$U_{be1} = nf * 0.026 * ln(\dfrac{I_{k1}}{Is})$$

$$U_{be2} = nf * 0.026 * ln(\dfrac{I_{k2}}{Is})$$

$$58 * 10^{-2} = nf * 0.026* ln(\dfrac{400 * 10^{-3}}{Is})$$

$$71 * 10^{-2} = nf * 0.026* ln(\dfrac{8420 * 10^{-3}}{Is})$$

$$Is = 4.99 * 10^{10}$$

$$nf = 1.64$$

$$U_{be3} = nf * 0.026 * ln(\dfrac{I_{k3}}{Is} + 1) + I_{b1} * Rb$$

$$Rb = 108 * 10^4$$

$$Bf = max(\dfrac{I_{k3}}{I_b})$$

Максимум находится при напряжении базы 0,71В

$$Bf = \dfrac{8420}{855} = 12.85$$

\section{Схемы для компьютерного расчета входных и выходных  ВАХ из программы моделирования}

\begin{figure}[H]
	\centering
	\includegraphics[width=0.7\linewidth]{images/shema1}
	\caption{Схема в LTSpice для расчета входных ВАХ}
	\label{fig:shema1}
\end{figure}

\begin{figure}[H]
	\centering
	\includegraphics[width=0.7\linewidth]{images/shema2}
	\caption{Схема в LTSpice для расчета выходных ВАХ}
	\label{fig:shema2}
\end{figure}

\section{Графики входных и выходных ВАХ}

\subsection{графики входных ВАХ}
\begin{figure}[H]
	\centering
	\includegraphics[width=0.7\linewidth]{images/graf8}
	\caption{Зависимость Ic}
	\label{fig:graf8}
\end{figure}


\begin{figure}[H]
	\centering
	\includegraphics[width=0.7\linewidth]{images/graf9}
	\caption{Зависимость Ib}
	\label{fig:graf9}
\end{figure}

\subsection{графики выходных ВАХ}

\begin{figure}[H]
	\centering
	\includegraphics[width=0.7\linewidth]{images/graf10}
	\caption{}
	\label{fig:graf10}
\end{figure}



\end{document} % конец документа

