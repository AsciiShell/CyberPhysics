%!TEX TS-program = xelatex

% Шаблон документа LaTeX создан в 2018 году
% Алексеем Подчезерцевым
% В качестве исходных использованы шаблоны
% 	Данилом Фёдоровых (danil@fedorovykh.ru) 
%		https://www.writelatex.com/coursera/latex/5.2.2
%	LaTeX-шаблон для русской кандидатской диссертации и её автореферата.
%		https://github.com/AndreyAkinshin/Russian-Phd-LaTeX-Dissertation-Template

\documentclass[a4paper,14pt]{article}

\input{data/preambular.tex}
\begin{document} % конец преамбулы, начало документа
\begin{titlepage}
	\begin{center}
		ФЕДЕРАЛЬНОЕ  ГОСУДАРСТВЕННОЕ АВТОНОМНОЕ \\
		ОБРАЗОВАТЕЛЬНОЕ УЧРЕЖДЕНИЕ ВЫСШЕГО ОБРАЗОВАНИЯ\\
		«НАЦИОНАЛЬНЫЙ ИССЛЕДОВАТЕЛЬСКИЙ УНИВЕРСИТЕТ\\
		«ВЫСШАЯ ШКОЛА ЭКОНОМИКИ»
	\end{center}
	
	\begin{center}
		\textbf{Московский институт электроники и математики}
		
		\textbf{Им. А.Н.Тихонова НИУ ВШЭ}
		
		\textbf{Департамент электронной инженерии}
	\end{center}	
	\vspace{5ex}
	\begin{center}
\textbf{<<ПОСТРОЕНИЕ, ПРИМЕНЕНИЕ И ИССЛЕДОВАНИЕ КОМПОНЕНТОВ СИСТЕМ СБОРА ДАННЫХ>>}
	\end{center}	
	\vspace{1ex}
	\begin{center}
\textbf{Отчёт по части 3 лабораторного практикума по дисциплине \\
	<<Электротехника, электроника и метрология>>, раздел <<Метрология>>(ЛР 8-9)}
	\end{center}	
	\vspace{5ex}
	
	\begin{multicols}{2}
	\vfill\null
	\columnbreak
	ВЫПОЛНИЛИ:
	
	Подчезерцев Алексей Евгеньевич
	
	Солодянкин Андрей Александрович
	
	группа БИВ172
	\end{multicols}

	\vfill
	\begin{center}
		Москва \the\year
	\end{center}
\end{titlepage}
\tableofcontents
\pagebreak
\section{Цель работы}

\begin{itemize}
	\item Изучение статических вольт-амперных характеристик МДП транзистора;
	\item Изучение способов и определение параметров его модели для схемотехнических расчетов
	\item Приобретение навыков работы с полупроводниковыми приборами;
	\item Приобретение навыков расчета схем с помощью SРIСЕ-симулятора.
\end{itemize}

\section{Теоретическое введение}

\subsection{Общие сведения}

МДП-транзистор - одна из разновидностей униполярного (полевого)
транзистора. Суть любого полевого транзистора состоит в следующем : в его
основе лежит проводящий канал с двумя контактами, называемыми сток
и исток; концентрация же носителей заряда в канале (а значит и электрическая
проводимость канала) управляется электрическим полем, возникающим
при подаче напряжения на третий вывод, называемый затвором. Отсюда
и название всех полевых транзисторов.

Особенностью МДП-транзисторов является то, что в них затвор отделён
от проводящего канала слоем диэлектрика, так что управляющая структура
такого транзистора составлена из слоёв Металла - Диэлектрика - Полупроводника
(МДП). МДП-транзистор, таким образом, - это полевой транзистор
с изолированным затвором - он представляет собой полупроводниковый
прибор, работа которого основана на использовании эффекта поля в структуре
металл-диэлектрик-полупроводник. В том распространённом частном
случае, когда в качестве диэлектрика применяется оксид кремния SiO 2, используется
название МОП-транзистор (МОПТ, Metal-Oxide-Semiconductor 
Field Effect Transistor - MOSFET). В современных транзисторах вместо SiO 2
нередко используются другие материалы с большей диэлектрической проницаемостью
(на основе гафния, циркония, тантала и др.), а в качестве материала
затвора поликремний, однако закрепившееся название «МОП-транзистор»
распространяется и на эти структуры.

Работа всех униполярных транзисторов, и в том числе МОПТ, основана
на использовании только одного типа носителей - основных (или электронов,
или дырок); процессы инжекции и диффузии в таких транзисторах не играют
принципиальной роли. Основным способом движения носителей является
дрейф в электрическом поле. Отметим, что в современных нанометровых
МОП-транзисторах подзатворный диэлектрик настолько тонкий, что сквозь
него возможно туннелирование носителей, так что в этом случае становится
заметным ток затвора.

\subsection{Классификация}

Первый признак классификации МОП-транзисторов - тип заряда в канале~
или тип проводимости канала. Рабочими носителями заряда в канале
транзистора могут являться электроны - такая разновидность называется
n-канальный МОПТ, области стока, истока и канала в нём имеют тип проводимости
n+, а подложка р-. В случае р-канального МОПТ рабочие носители
заряда дырки, области стока, истока и канала имеют тип проводимости р+,
а подложка n-. При одинаковой конструкции n- и р-канальный транзисторы
имеют противоположные по знаку управляющие напряжения и выходные то ки.
Второй признак классификации - наличие/отсутствие структурно выраженного
канала при нулевом напряжении затвора. При производстве транзистора
в области между стоком и истоком создаётся мелкая примесная область,
насыщенная основными или же неосновными носителями заряда; это
позволяет определить для транзистора напряжение, при котором он открывается
(что означает, образуется проводящий канал). В современных схемах
высокого уровня сложности нередко используются МОПТ, открытые в разной
степени (т. е. в их приповерхностные области внедрена примесь различного
типа проводимости и различной концентрации); это позволяет более
тонко регулировать состояние схемы и снижать энергопотребление. Два
крайних типа структуры: 1) структура МОПТ, которая достаточно сильно открыта
в отсутствие напряжения затвора (\imref{fig:teormop1} a); 2) структура МОПТ, которая
закрыта в отсутствие напряжения затвора (\imref{fig:teormop1} б) - называются, соответственно,
МОПТ со встроенным (или собственным) каналом и МОПТ с
наведённым (или индуцированным) каналом; схемы их управления различаются.
Обозначения различных типов МОПТ на электрических схемах приведены
на \imref{fig:teormop2}.

\begin{figure}
	\centering
	\includegraphics[width=\linewidth]{image/Teor_MOP_1}
	\caption{Структуры р-канальных МОП-транзисторов со встроенным (а) и индуцированным (6) каналами}
	\label{fig:teormop1}
\end{figure}

\begin{figure}
	\centering
	\includegraphics[width=0.7\linewidth]{image/Teor_MOP_2}
	\caption{Обозначения МОПТ на электрических схемах}
	\label{fig:teormop2}
\end{figure}

\subsection{Принцип действия МОПТ}

Принцип действия МОПТ с наведённым каналом. Работа МОПТ
определяется в основном двумя управляющими напряжениями: затвора
и стока, которые обычно отсчитываются относительно потенциала истока:
Vзи, Vси (\imref{fig:teormop3}).
При изменении напряжения затвора Vзи формируется (или наоборот
перекрывается) проводящий канал. При Vзи = О канал отсутствует, ток в выходной
цепи равен обратному тепловому току р-n-переходов.
При повышении Vзи сначала образуется обеднённый слой ( объёмный заряд
ионов: акцепторов в случае n-МОПТ и доноров в случае р-МОПТ),
 а затем - при напряжении затвора Vзи = $Vpor$, называемом пороговым,
- инверсионный слой электронов, который как раз и является проводящим
каналом. Свободные носители заряда, составляющие канал,
притягиваются электрическим полем затвора частично из области подложки,
но большей частью из областей стока и истока. На сток-затворной
( ещё говорят: передаточной, или проходной 1 ) вольт-амперной характеристике
Jс(Vзи) значение Vзи = Vпор разделяет два участка (\imref{fig:teormop5} а) . Согласно
физическому определению, пороговым напряжением затвора называется такое,
при котором концентрация свободных электронов в тонком приповерхностном
слое становится равной исходной концентрации свободных дырок в
объёме подложки. При Vзи > Vпор транзистор условно считается открытым.
При изменении напряжения стока меняется форма канала. При Vси > О
потенциал в канале является неравномерным: вблизи истока (х = О, если отсчитывать
от начала канала) он определяется практически только полем затвора
и равен Vзи - Vпор, а вблизи стока (х = L, где L - длина затвора) совместным
действием полей затвора и стока и равен Vзи - f!;.юр - Vси. Соответственно,
с увеличением Vси толщина канала со стороны стока уменьшается.
 При достижении напряжением стока некоторого критического
значения, называемого напряжением насыщения: Vси,нас = Vзи - Vпор, сечение
канала вблизи стока в точке х = L уменьшается до О (это называется отсечкой
канала), так как напряжение между затвором и поверхностью полупроводника
в этой точке становится равным пороговому напряжению.
При дальнейшем увеличении напряжения стока Vси > Vси,нас фактическая
(«эффективная») длина канала LэФФ = L - ЛL становится меньше L
2, а оставшееся до области стока расстояние ЛL занимает расширившаяся
обеднённая область (ОПЗ) обратносмещённого стокового р-nперехода.
Проводимость транзистора обеспечивается следующим образом:
носители заряда, прошедшие из области истока к концу канала, подхватываются
сильным электрически полем стокового р-n-перехода и дрейфуют к области
стока. В итоге ток стока слабо увеличивается, график его
зависимости от напряжения стока имеет небольшой наклон. Геометрическое
место значений Vси,нас = Vзи - Vпор разделяет триодный (крутой) и пентодный
(пологий) участки на выходных характеристиках (\imref{fig:teormop4}).
Принцип действия МОПТ со встроенным каналом отличается тем,
что пороговое напряжение у такого прибора (называемое здесь напряжением
отсечки) имеет отрицательное значение, что влияет на вид его волыамперных
характеристик.

\begin{figure}
	\centering
	\includegraphics[width=\linewidth]{image/Teor_MOP_3}
	\caption{Общий вид (а) и продольный разрез (6) n-канального МОП-транзистора}
	\label{fig:teormop3}
\end{figure}

\begin{figure}
	\centering
	\includegraphics[width=\linewidth]{image/Teor_MOP_4}
	\caption{Вид сток-затворных Iс(Vзи) (а) и выходных Iс(Vси) (6) вольт- амперных характеристик МОПТ}
	\label{fig:teormop4}
\end{figure}

\section{Схемы измерений}

\section{Результаты измерений}

\begin{table}[H]
	\begin{center}
		\tablecaption{Сток-затворная ВАХ транзистора при $V_c = -0.1$В}
		\begin{tabular}{|l|l|}
			\hline
			$V_{zi}$, V & $I_c$, мкA \\ \hline
			-6.2	&	-13.6\\ \hline
			-4.8	&	-13.2\\ \hline
			-4.3	&	-12.7\\ \hline
			-4.1	&	-12.2\\ \hline
			-3.9	&	-11.6\\ \hline
			-3.8	&	-11.4\\ \hline
			-3.7	&	-10.4\\ \hline
			-3.6	&	-10.0\\ \hline
			-3.5	&	-9.0\\ \hline
			-3.5	&	-8.6\\ \hline
			-3.42	&	-7.4\\ \hline
			-3.36	&	-6.5\\ \hline
			-3.2	&	-2.4\\ \hline
			-3.0	&	0.0\\ \hline
			-2.6	&	0.7\\ \hline
		\end{tabular}
	\end{center}
\end{table}

\begin{table}[H]
	\begin{center}
		\tablecaption{Входная ВАХ стока транзистора при $V_z = -3.5$В}
		\begin{tabular}{|l|l|}
			\hline
			$V_{ci}$, V & $I_c$, мкA \\ \hline
			-10.4	&	-63 \\ \hline
			-9	&	-60 \\ \hline
			-8.2	&	-57 \\ \hline
			-7.7	&	-55.3 \\ \hline
			-4.7	&	-49.8 \\ \hline
			-3.5	&	-47.8 \\ \hline
			-3.1	&	-46.9 \\ \hline
			-2.2	&	-45.1 \\ \hline
			-2	&	-44.4 \\ \hline
			-1.7	&	-43.2 \\ \hline
			-0.8	&	-39 \\ \hline
			-0.5	&	-37 \\ \hline
			-0.4	&	-36 \\ \hline
			-0.2	&	-30.7 \\ \hline
			-0.02	&	-18.9 \\ \hline
			-0.1	&	-5 \\ \hline
		\end{tabular}
	\end{center}
\end{table}

\begin{table}[H]
	\begin{center}
		\tablecaption{Входная ВАХ стока транзистора при $V_z = -4.3$В}
		\begin{tabular}{|l|l|}
			\hline
			$V_{ci}$, V & $I_c$, мкA \\ \hline
			-10.18	&	-409 \\ \hline
			-9.9	&	-407 \\ \hline
			-9	&	-394 \\ \hline
			-7.58	&	-382 \\ \hline
			-4.48	&	-348 \\ \hline
			-3.02	&	-326 \\ \hline
			-1.86	&	-305 \\ \hline
			-1.5	&	-298 \\ \hline
			-0.78	&	-259 \\ \hline
			-0.47	&	-223 \\ \hline
			-0.43	&	-214 \\ \hline
			-0.31	&	-180 \\ \hline
			-0.25	&	-147 \\ \hline
			-0.14	&	-92 \\ \hline
			-0.08	&	-57 \\ \hline
			-0.06	&	-42 \\ \hline
			-0.01	&	-4.8 \\ \hline
		\end{tabular}
	\end{center}
\end{table}

\begin{table}[H]
	\begin{center}
		\tablecaption{Входная ВАХ стока транзистора при $V_z = -4.92$В}
		\begin{tabular}{|l|l|}
			\hline
			$V_{ci}$, V & $I_c$, мкA \\ \hline
			-7.9	&	-814 \\ \hline
			-6.5	&	-787 \\ \hline
			-5.15	&	-759 \\ \hline
			-3.63	&	-719 \\ \hline
			-1.6	&	-636 \\ \hline
			-0.77	&	-530 \\ \hline
			-0.57	&	-455 \\ \hline
			-0.42	&	-372 \\ \hline
			-0.34	&	-312 \\ \hline
			-0.25	&	-238 \\ \hline
			-0.2	&	-197 \\ \hline
			-0.16	&	-160 \\ \hline
			-0.09	&	-93.1 \\ \hline
			-0.08	&	-82.3 \\ \hline
			-0.07	&	-72.3 \\ \hline
			-0.06	&	-60.6 \\ \hline
			-0.05	&	-58.5 \\ \hline
			-0.04	&	-42.4 \\ \hline
			0	&	-5.7 \\ \hline
		\end{tabular}
	\end{center}
\end{table}

\subsection{Графики измерений}

\begin{figure}[H]
	\centering
	\includegraphics[width=0.7\linewidth]{image/graph_1}
	\caption{Сток-затворная ВАХ транзистора}
	\label{fig:graph1}
\end{figure}

\begin{figure}[H]
	\centering
	\includegraphics[width=0.7\linewidth]{image/graph_2}
	\caption{Входные ВАХ стока транзистора}
	\label{fig:graph2}
\end{figure}

\section{Вычисление параметров модели}
\section{Схемы расчета ВАХ}
\section{Графики ВАХ}
\end{document} % конец документа