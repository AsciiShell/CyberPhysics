%!TEX TS-program = xelatex

% Шаблон документа LaTeX создан в 2018 году
% Алексеем Подчезерцевым
% В качестве исходных использованы шаблоны
% 	Данилом Фёдоровых (danil@fedorovykh.ru) 
%		https://www.writelatex.com/coursera/latex/5.2.2
%	LaTeX-шаблон для русской кандидатской диссертации и её автореферата.
%		https://github.com/AndreyAkinshin/Russian-Phd-LaTeX-Dissertation-Template

\documentclass[a4paper,14pt]{article}

\input{data/preambular.tex}
\begin{document} % конец преамбулы, начало документа
\begin{titlepage}
	\begin{center}
		ФЕДЕРАЛЬНОЕ  ГОСУДАРСТВЕННОЕ АВТОНОМНОЕ \\
		ОБРАЗОВАТЕЛЬНОЕ УЧРЕЖДЕНИЕ ВЫСШЕГО ОБРАЗОВАНИЯ\\
		«НАЦИОНАЛЬНЫЙ ИССЛЕДОВАТЕЛЬСКИЙ УНИВЕРСИТЕТ\\
		«ВЫСШАЯ ШКОЛА ЭКОНОМИКИ»
	\end{center}
	
	\begin{center}
		\textbf{Московский институт электроники и математики}
		
		\textbf{Им. А.Н.Тихонова НИУ ВШЭ}
		
		\textbf{Департамент электронной инженерии}
	\end{center}	
	\vspace{5ex}
	\begin{center}
\textbf{<<ПОСТРОЕНИЕ, ПРИМЕНЕНИЕ И ИССЛЕДОВАНИЕ КОМПОНЕНТОВ СИСТЕМ СБОРА ДАННЫХ>>}
	\end{center}	
	\vspace{1ex}
	\begin{center}
\textbf{Отчёт по части 3 лабораторного практикума по дисциплине \\
	<<Электротехника, электроника и метрология>>, раздел <<Метрология>>(ЛР 8-9)}
	\end{center}	
	\vspace{5ex}
	
	\begin{multicols}{2}
	\vfill\null
	\columnbreak
	ВЫПОЛНИЛИ:
	
	Подчезерцев Алексей Евгеньевич
	
	Солодянкин Андрей Александрович
	
	группа БИВ172
	\end{multicols}

	\vfill
	\begin{center}
		Москва \the\year
	\end{center}
\end{titlepage}

\section{Резонанс напряжений в сети синусоидального тока}
\subsection{вычисление резонансной частоты}
$C =  0.1$ мкФ 

$L = 10$ мГн

$$\nu_r = \dfrac{1}{2*\pi * \sqrt{L*C}} =\dfrac{1}{2*\pi* \sqrt{10 * 10^{-3}*0.1*10^{-6}}} = 5032.92121045 \approx 5000 $$

\begin{figure}[H]
	\centering
	\includegraphics[width=0.5\linewidth]{shem.png}
	\caption{Схема RCL цепи}	
\end{figure}


\subsection{1 случай}
$R_1 = 50$ Ом 

$C =  0.1$ мкФ 

$L = 10$ мГн

$R_L = 28$ Ом

\subsubsection{модуляция процесса в программе SPICE}

\begin{figure}[H]
	\centering
	\includegraphics[width=0.5\linewidth]{shem1.png}	
\end{figure}

\begin{figure}[H]
	\centering
	\includegraphics[width=0.9\linewidth]{graf_3_V123.png}
	\caption{показания вольтметров}	
\end{figure}

\begin{figure}[H]
	\centering
	\includegraphics[width=0.9\linewidth]{graf_3_i.png}
	\caption{показания амперметра}	
\end{figure}

\subsubsection{построение графиков по измеренным данным}

\begin{figure}[H]
	\centering
	\includegraphics[width=0.9\linewidth]{mat_3_V123.png}
	\caption{показания вольтметров}	
\end{figure}

\begin{figure}[H]
	\centering
	\includegraphics[width=0.9\linewidth]{mat_1_i.png}
	\caption{показания амперметра}	
\end{figure}


\begin{figure}[H]
	\centering
	\includegraphics[width=0.9\linewidth]{mat_1_fi.png}
		\caption{сдвиг фаз между током и напряжением}	
	\end{figure}


\subsection{2 случай}
$R_1 = 200$ Ом 

$C =  0.1$ мкФ 

$L = 10$ мГн

$R_L = 28$ Ом

\subsubsection{модуляция процесса в программе SPICE}

\begin{figure}[H]
	\centering
	\includegraphics[width=0.5\linewidth]{shem2.png}	
\end{figure}

\begin{figure}[H]
	\centering
	\includegraphics[width=0.9\linewidth]{graf_4_V123.png}
	\caption{показания вольтметров}	
\end{figure}

\begin{figure}[H]
	\centering
	\includegraphics[width=0.9\linewidth]{graf_4_i.png}
	\caption{показания амперметра}	
\end{figure}

\subsubsection{построение графиков по измеренным данным}

\begin{figure}[H]
	\centering
	\includegraphics[width=0.9\linewidth]{mat_4_V123.png}
	\caption{показания вольтметров}	
\end{figure}

\begin{figure}[H]
	\centering
	\includegraphics[width=0.9\linewidth]{mat_2_i.png}
	\caption{показания амперметра}	
\end{figure}


\begin{figure}[H]
	\centering
	\includegraphics[width=0.9\linewidth]{mat_2_fi.png}
	\caption{сдвиг фаз между током и напряжением}	
\end{figure}


\subsection{Контрольные вопросы}
$$\nu_r = \dfrac{1}{2*\pi * \sqrt{L*C}}$$
\begin{itemize}
	\item При увеличении $C$ резонансная частота при РН уменьшается.
	\item При увеличении $L$ резонансная частота при РН уменьшается.
\end{itemize}
\end{document} % конец документа