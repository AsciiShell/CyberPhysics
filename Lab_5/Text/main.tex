%!TEX TS-program = xelatex

% Шаблон документа LaTeX создан в 2018 году
% Алексеем Подчезерцевым
% В качестве исходных использованы шаблоны
% 	Данилом Фёдоровых (danil@fedorovykh.ru) 
%		https://www.writelatex.com/coursera/latex/5.2.2
%	LaTeX-шаблон для русской кандидатской диссертации и её автореферата.
%		https://github.com/AndreyAkinshin/Russian-Phd-LaTeX-Dissertation-Template

\documentclass[a4paper,14pt]{article}

\input{data/preambular.tex}
\begin{document} % конец преамбулы, начало документа
\begin{titlepage}
	\begin{center}
		ФЕДЕРАЛЬНОЕ  ГОСУДАРСТВЕННОЕ АВТОНОМНОЕ \\
		ОБРАЗОВАТЕЛЬНОЕ УЧРЕЖДЕНИЕ ВЫСШЕГО ОБРАЗОВАНИЯ\\
		«НАЦИОНАЛЬНЫЙ ИССЛЕДОВАТЕЛЬСКИЙ УНИВЕРСИТЕТ\\
		«ВЫСШАЯ ШКОЛА ЭКОНОМИКИ»
	\end{center}
	
	\begin{center}
		\textbf{Московский институт электроники и математики}
		
		\textbf{Им. А.Н.Тихонова НИУ ВШЭ}
		
		\textbf{Департамент электронной инженерии}
	\end{center}	
	\vspace{5ex}
	\begin{center}
\textbf{<<ПОСТРОЕНИЕ, ПРИМЕНЕНИЕ И ИССЛЕДОВАНИЕ КОМПОНЕНТОВ СИСТЕМ СБОРА ДАННЫХ>>}
	\end{center}	
	\vspace{1ex}
	\begin{center}
\textbf{Отчёт по части 3 лабораторного практикума по дисциплине \\
	<<Электротехника, электроника и метрология>>, раздел <<Метрология>>(ЛР 8-9)}
	\end{center}	
	\vspace{5ex}
	
	\begin{multicols}{2}
	\vfill\null
	\columnbreak
	ВЫПОЛНИЛИ:
	
	Подчезерцев Алексей Евгеньевич
	
	Солодянкин Андрей Александрович
	
	группа БИВ172
	\end{multicols}

	\vfill
	\begin{center}
		Москва \the\year
	\end{center}
\end{titlepage}

\section{Резонанс напряжений в сети синусоидального тока}
\subsection{вычисление резонансной частоты}
$C =  0.1$ мкФ 

$L = 10$ мГн

$$\nu_r = \dfrac{1}{2*\pi * \sqrt{L*C}} =\dfrac{1}{2*\pi* \sqrt{10 * 10^{-3}*0.1*10^{-6}}} = 5032.92121045 \approx 5000 $$

\subsection{1}
$R_1 = 50$ Ом 

$C =  0.1$ мкФ 

$L = 10$ мГн

$R_L = 28$ Ом
\begin{figure}[H]
	\centering
	\includegraphics[width=0.5\linewidth]{shem.png}
	\caption{Схема RCL цепи}	
\end{figure}


\subsection{модуляция процесса в программе SPICE}

\begin{figure}[H]
	\centering
	\includegraphics[width=0.5\linewidth]{shem1.png}	
\end{figure}

\begin{figure}[H]
	\centering
	\includegraphics[width=0.9\linewidth]{graf_3_V123.png}
	\caption{показания вольтметров}	
\end{figure}

\begin{figure}[H]
	\centering
	\includegraphics[width=0.9\linewidth]{graf_3_i.png}
	\caption{показания амперметра}	
\end{figure}

\end{document} % конец документа



\begin{figure}[H]
	\centering
	\includegraphics[width=0.5\linewidth]{scema_RC.jpg}
	\caption{Схема RC цепи}	
\end{figure}


\begin{figure}[H]
	\centering
	\includegraphics[width=\linewidth]{1.png}
	\includegraphics[width=\linewidth]{graph1.png}
	\caption{100 нФ, 100 Ом}	
\end{figure}

\begin{figure}[H]
	\centering
	\includegraphics[width=\linewidth]{2.png}
	\includegraphics[width=\linewidth]{graph2.png}
	\caption{100 нФ, 100 Ом}	
\end{figure}

\begin{figure}[H]
	\centering
	\includegraphics[width=\linewidth]{3.png}
	\includegraphics[width=\linewidth]{graph3.png}
	\caption{22 нФ, 400 Ом}	
\end{figure}

\begin{figure}[H]
	\centering
	\includegraphics[width=\linewidth]{4.png}
	\includegraphics[width=\linewidth]{graph4.png}
	\caption{22 нФ, 50 Ом}	
\end{figure}

\subsection{Контрольные вопросы}
\begin{itemize}
	\item $1 - e^{-\dfrac{\tau}{\tau}} = 0.632$
	\item $\tau = RC$. При увеличении емкости конденсатора длительность переходного процесса увеличивается.
\end{itemize}


\section{Переходные процессы в RL цепях}

\begin{figure}[H]
	\centering
	\includegraphics[width=0.5\linewidth]{scema_RL.jpg}
	\caption{Схема RL цепи}	
\end{figure}


\begin{figure}[H]
	\centering
	\includegraphics[width=\linewidth]{5.png}
	\includegraphics[width=\linewidth]{graph5.png}
	\caption{1 мГ, 100 Ом}	
\end{figure}

\begin{figure}[H]
	\centering
	\includegraphics[width=\linewidth]{6.png}
	\includegraphics[width=\linewidth]{graph6.png}
	\caption{1 мГ, 100 Ом}	
\end{figure}

\begin{figure}[H]
	\centering
	\includegraphics[width=\linewidth]{7.png}
	\includegraphics[width=\linewidth]{graph7.png}
	\caption{10 мГ, 100 Ом}	
\end{figure}

\begin{figure}[H]
	\centering
	\includegraphics[width=\linewidth]{8.png}
	\includegraphics[width=\linewidth]{graph8.png}
	\caption{10 мГ, 400 Ом}	
\end{figure}

\subsection{Контрольные вопросы}
\begin{itemize}
	\item $t \sim \tau.$ При увеличении $\tau$ длительность переходного процесса увеличивается.
	\item $\tau = \dfrac{L}{R}$. При уменьшении индуктивности катушки длительность переходного процесса уменьшается.
\end{itemize}
