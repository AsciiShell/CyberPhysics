%!TEX TS-program = xelatex

% Шаблон документа LaTeX создан в 2018 году
% Алексеем Подчезерцевым
% В качестве исходных использованы шаблоны
% 	Данилом Фёдоровых (danil@fedorovykh.ru) 
%		https://www.writelatex.com/coursera/latex/5.2.2
%	LaTeX-шаблон для русской кандидатской диссертации и её автореферата.
%		https://github.com/AndreyAkinshin/Russian-Phd-LaTeX-Dissertation-Template

\documentclass[a4paper,14pt]{article}

\input{data/preambular.tex}
\begin{document} % конец преамбулы, начало документа
\begin{titlepage}
	\begin{center}
		ФЕДЕРАЛЬНОЕ  ГОСУДАРСТВЕННОЕ АВТОНОМНОЕ \\
		ОБРАЗОВАТЕЛЬНОЕ УЧРЕЖДЕНИЕ ВЫСШЕГО ОБРАЗОВАНИЯ\\
		«НАЦИОНАЛЬНЫЙ ИССЛЕДОВАТЕЛЬСКИЙ УНИВЕРСИТЕТ\\
		«ВЫСШАЯ ШКОЛА ЭКОНОМИКИ»
	\end{center}
	
	\begin{center}
		\textbf{Московский институт электроники и математики}
		
		\textbf{Им. А.Н.Тихонова НИУ ВШЭ}
		
		\textbf{Департамент электронной инженерии}
	\end{center}	
	\vspace{5ex}
	\begin{center}
\textbf{<<ПОСТРОЕНИЕ, ПРИМЕНЕНИЕ И ИССЛЕДОВАНИЕ КОМПОНЕНТОВ СИСТЕМ СБОРА ДАННЫХ>>}
	\end{center}	
	\vspace{1ex}
	\begin{center}
\textbf{Отчёт по части 3 лабораторного практикума по дисциплине \\
	<<Электротехника, электроника и метрология>>, раздел <<Метрология>>(ЛР 8-9)}
	\end{center}	
	\vspace{5ex}
	
	\begin{multicols}{2}
	\vfill\null
	\columnbreak
	ВЫПОЛНИЛИ:
	
	Подчезерцев Алексей Евгеньевич
	
	Солодянкин Андрей Александрович
	
	группа БИВ172
	\end{multicols}

	\vfill
	\begin{center}
		Москва \the\year
	\end{center}
\end{titlepage}
\tableofcontents
\pagebreak

\section{ЦЕЛИ РАБОТЫ}
Целями данной работы являются:

\begin{itemize}
	\item получение навыков получения, обработки и представления результатов многократных измерений;
	\item формирование базовых навыков работы в среде NI LabVIEW по раз-работке компонентов для автоматизированной обработки результатов однократных измерений.
\end{itemize}

\section{ПРИМЕНЯЕМОЕ ОБОРУДОВАНИЕ И ПРОГРАММНОЕ ОБЕСПЕЧЕНИЕ}

\begin{enumerate}
	\item	Персональный компьютер (ПК).
	\item	Плата сбора данных NI-DAQ M-series PCI-6221/6251.
	\item	Кабель NI SH68-68-EPM Shielded Cable.
	\item	Коннекторный блок NI BNC 2120 
	\item	Провод соединительный коаксиальный (BNC-BNC).
	\item	Среда разработки NI LabVIEW 2013 Professional.
\end{enumerate}


\section{СТРУКТУРА ПРИКЛАДНОГО 	ПРОГРАММНОГО ОБЕСПЕЧЕНИЯ}
\subsection{Виртуальный прибор сбора и визуализации данных}

Прибор, изображенный на Рис. \ref{img:av_vi} позволяет измерять напряжение на некотором элемента.
В качестве источника используется блок DAQ Assistant (Рис. \ref{img:av_schema}).

С помощью панели управления можно задать задержку между соседними снятиями сигналов, а так же остановить вычисления.
Снятые значения выводятся на диаграмму в виде точек.
Кроме того, выполняется обработка данных и усреднение результатов за последние 4 измерения, что сглаживает распределение величин.

В конечном счёте пользователь получает два графика -- текущее значение, которое отмечено точками, и усредненное в процессе измерения, обозначенное красной линией.

\begin{figure}[H]
	\centering		
	\includegraphics[width=\linewidth]{image/av_vi}
	\caption{Передняя панель Acquire Voltage}\label{img:av_vi}
\end{figure}


\begin{figure}[H]
	\centering		
	\includegraphics[width=\linewidth]{image/av_schema}
	\caption{Блок-схема Acquire Voltage}\label{img:av_schema}
\end{figure}

\subsection{Виртуальный прибор выполнения многократных измерений }

Краткое описание ВП Voltage Multiple Measurements.vi (ЛР 5, упр.5) со ссылками на рисунки передней панели и блок схемы. Рисунки передней панели и блок схемы. Описание не должно дублировать текст МУ.

\subsection{Some text}
Разделы с кратким описанием ВП Grubbs\_Test.vi или Outlier Test.vi (ЛР 6), mult\_process.vi (ЛР 7) со ссылками на рисунки передних панелей и блок-схем. Рисунки передних панелей и блок-схем. Можно приводить не все кейсы, а только те, на которых выполняется основная часть обработки. Описание не должно дублировать текст МУ 

\section{ПОРЯДОК ВЫПОЛНЕНИЯ РАБОТЫ}
\begin{enumerate}
\item 	Построение ВП для сбора и визуализации данных на основе Express-VI DAQ Assistant.
 Выполнение сбора и визуализации данных. 
 Исследование режимов отображения данных на диаграмме Waveform Chart. 
 Экспорт измеренных значений (таблица 1).
\item 	Построение ВП для выполнения и сохранения результатов многократных измерений.
 Выполнение многократных измерений, сохранение полученных результатов (таблица 2).
\item 	Построение ВП для проверки результатов измерений на наличие промахов. 
Проверка тестовой выборки (таблица 3.1), проверка результатов измерений (таблица  3.2).
\item 	Построение ВП для обработки результатов многократных измерений. 
Обработка тестовой выборки (таблица 4.1), результатов измерений (таблица 4.2).
\end{enumerate}

\section{ РЕЗУЛЬТАТЫ ИЗМЕРЕНИЙ И ВЫЧИСЛЕНИЙ}

\subsection{Результаты сбора данных}
Таблица и график с результатами измерений, полученные с помо-щью Acquire Voltage.vi (ЛР 5). 
Таблица 1
\subsection{Результаты многократных измерений}

Таблица с результатами измерений, полученных с помощью Voltage Multiple Measurements.vi (ЛР 6). Также записать максимальное, мини-мальное и среднее значения. 
Таблица 2
\subsection{Результаты проверки на промахи по критерию Граббса}

\section{ВЫВОДЫ ПО РАБОТЕ}
\end{document} % конец документа