%!TEX TS-program = xelatex

% Шаблон документа LaTeX создан в 2018 году
% Алексеем Подчезерцевым
% В качестве исходных использованы шаблоны
%     Данилом Фёдоровых (danil@fedorovykh.ru)
%        https://www.writelatex.com/coursera/latex/5.2.2
%    LaTeX-шаблон для русской кандидатской диссертации и её автореферата.
%        https://github.com/AndreyAkinshin/Russian-Phd-LaTeX-Dissertation-Template

\documentclass[a4paper,14pt]{article}

\input{data/preambular.tex}
\begin{document} % конец преамбулы, начало документа
\begin{titlepage}
	\begin{center}
		ФЕДЕРАЛЬНОЕ  ГОСУДАРСТВЕННОЕ АВТОНОМНОЕ \\
		ОБРАЗОВАТЕЛЬНОЕ УЧРЕЖДЕНИЕ ВЫСШЕГО ОБРАЗОВАНИЯ\\
		«НАЦИОНАЛЬНЫЙ ИССЛЕДОВАТЕЛЬСКИЙ УНИВЕРСИТЕТ\\
		«ВЫСШАЯ ШКОЛА ЭКОНОМИКИ»
	\end{center}
	
	\begin{center}
		\textbf{Московский институт электроники и математики}
		
		\textbf{Им. А.Н.Тихонова НИУ ВШЭ}
		
		\textbf{Департамент электронной инженерии}
	\end{center}	
	\vspace{5ex}
	\begin{center}
\textbf{<<ПОСТРОЕНИЕ, ПРИМЕНЕНИЕ И ИССЛЕДОВАНИЕ КОМПОНЕНТОВ СИСТЕМ СБОРА ДАННЫХ>>}
	\end{center}	
	\vspace{1ex}
	\begin{center}
\textbf{Отчёт по части 3 лабораторного практикума по дисциплине \\
	<<Электротехника, электроника и метрология>>, раздел <<Метрология>>(ЛР 8-9)}
	\end{center}	
	\vspace{5ex}
	
	\begin{multicols}{2}
	\vfill\null
	\columnbreak
	ВЫПОЛНИЛИ:
	
	Подчезерцев Алексей Евгеньевич
	
	Солодянкин Андрей Александрович
	
	группа БИВ172
	\end{multicols}

	\vfill
	\begin{center}
		Москва \the\year
	\end{center}
\end{titlepage}
\tableofcontents
\pagebreak

\section{ЦЕЛИ РАБОТЫ}
Целями данной работы являются:

\begin{itemize}
    \item получение навыков получения, обработки и представления результатов многократных измерений;
    \item формирование базовых навыков работы в среде NI LabVIEW по разработке компонентов для автоматизированной обработки результатов однократных измерений.
\end{itemize}

\section{ПРИМЕНЯЕМОЕ ОБОРУДОВАНИЕ И ПРОГРАММНОЕ ОБЕСПЕЧЕНИЕ}

\begin{enumerate}
    \item    Персональный компьютер (ПК).
    \item    Плата сбора данных NI-DAQ M-series PCI-6221/6251.
    \item    Кабель NI SH68-68-EPM Shielded Cable.
    \item    Коннекторный блок NI BNC 2120
    \item    Провод соединительный коаксиальный (BNC-BNC).
    \item    Среда разработки NI LabVIEW 2013 Professional.
\end{enumerate}


\section{СТРУКТУРА ПРИКЛАДНОГО     ПРОГРАММНОГО ОБЕСПЕЧЕНИЯ}
\subsection{ВП сбора и обработки данных}

%ВП Data acquisition and analysis.vi (ЛР 08, упр. 2)

\subsection{ВП протоколирования результатов}

%Get and Write Data.vi (ЛР 08, упр. 3) 

\subsection{ВП измерения частоты и периода}

%DAQ Frequency Meter.vi (ЛР 09)

\subsection{ВП протоколирования результатов}

% same

\section{ПОРЯДОК ВЫПОЛНЕНИЯ РАБОТЫ}
\begin{enumerate}
	\item Проверка работоспособности оборудования в NI MAX. 
	\item Построение ВП для сбора и обработки данных на основе Express-VI.
	\item Построение ВП для формирования протокола. 
	\item Выполнение сбора и обработки данных. Получение измеренных значений амплитуды, частоты и коэффициента нелинейных искажений. Протоколирование измеренных значений (таблица 1).
	\item По аналогии с пп.1-4
\end{enumerate}

\section{РЕЗУЛЬТАТЫ ИЗМЕРЕНИЙ И ВЫЧИСЛЕНИЙ}

\subsection{Результаты сбора данных}

% Таблица 1 с измеренными значениями, сохранёнными в текстовом файле при выполнении ЛР 08, упр.3. Обоснование связи полученных амплитудного и среднеквадратичного значений

\subsection{Результаты измерения частоты и периода}

% Таблицы с измеренными значениями, сохранёнными в текстовом файле при выполнении задания ЛР 09 . Если задание не выполнялась, привести короткую таблицу, заполненную по данным со скриншотов

\section{ВЫВОДЫ ПО РАБОТЕ}

%Привести выводы по полученным результатам, а также отметить особенности использованных структур LabVIEW и возможностей устройства DAQ
В ходе работы были получены следующие результаты:

\begin{itemize}
	\item В данных работах были использованы алгоритмические структуры циклы while, for, оператор множественного выбора switch/case, условный оператор, тернарный оператор (select), константы.
	Для передачи данных от одной итерации цикла к последующим использовались сдвиговые регистры.
	\item Для реализации паттерна конечного автомата использовалась константа со списком всех состояний, цикл while, который останавливался при шаге завершения работы, структура case, в которой были определены действия для каждого случая и следующая итерация в зависимости от ситуации.
\end{itemize}

\end{document} % конец документа